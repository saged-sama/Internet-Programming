\chapter{Proposed System}
\section{Overview}
The project is the development of  E Lib that aims to improve the efficiency and accuracy of the library management process. The software will be used by the library staff and admin to manage books, users, borrowing, returning, notifications, and generate reports and analytics. The objectives of the project are to automate and streamline the library management process, reduce human error, and improve the overall user experience. The project scope includes the development of the software, testing, and implementation.

\section{Functional Requirements}

\begin{center}

% Different accounts
\begin{tabular}{ | m{5em} | m{6.5cm}|  } 
  \hline
  ID & 3.2.1\\ 
  \hline
  Name & User Management \\ 
  \hline
  Description & 
  \begin{enumerate}
      \item The system shall allow the admin to create and manage staff accounts.
      \item The system shall allow staff to update their account information.
      \item The system shall allow the admin to reset passwords for staff accounts.

  \end{enumerate} \\
  \hline
  
  Priority & High\\
  \hline 
  Reference & \\
  \hline
\end{tabular}
\end{center}

% Input income, expenses
\begin{center}
\begin{tabular}{ | m{5em} | m{6.5cm}|  } 
  \hline
  ID & 3.2.2\\ 
  \hline
  Name &  Book Management\\ 
  \hline
  Description & 
  \begin{enumerate}
      \item  The system shall allow the staff to add new books to the library database with information such as title, author, publication date, ISBN, and category.
      \item The system shall allow the staff to update book information such as book cover, title, author, and category.
      \item The system shall allow the staff to search for a book by title, author, ISBN, or category.
      \item The system shall allow the staff to delete a book from the library database.


  \end{enumerate} \\
  \hline
  
  Priority & High\\
  \hline
  Reference & \\
  \hline
\end{tabular}
\end{center}

% Planned payments and income
\begin{center}
\begin{tabular}{ | m{5em} | m{6.5cm}|  } 
  \hline
  ID & 3.2.3\\ 
  \hline
  Name & Borrowing Management \\ 
  \hline
  Description & 
  \begin{enumerate}
  
      \item The system shall allow the staff to borrow a book to a member.
      \item The system shall allow the staff to check if a book is available for borrowing.
      \item The system shall allow the staff to record the borrowing details such as the book ID, member ID, and borrowing date.
      \item The system shall send notifications to the member if the book is not returned by the due date.

  \end{enumerate} \\
  \hline
  
  Priority & High \\
  \hline
  Reference & \\
  \hline
\end{tabular}
\end{center}

% Due and lent tabs
\begin{center}
\begin{tabular}{ | m{5em} | m{6.5cm}|  } 
  \hline
  ID & 3.2.4\\ 
  \hline
  Name & Returning Management \\ 
  \hline
  Description & 
  \begin{enumerate}
      \item The system shall allow the staff to return a borrowed book.
      \item The system shall record the return details such as the book ID, member ID, and return date.
      \item The system shall calculate any late fees or fines for overdue books.
  \end{enumerate} \\
  \hline
  
  Priority & High\\
  \hline
  Reference & \\
  \hline
\end{tabular}
\end{center}

% password access
\begin{center}
\begin{tabular}{ | m{5em} | m{6.5cm}|  } 
  \hline
  ID & 3.2.5\\ 
  \hline
  Name & Reports and Analytics\\ 
  \hline
  Description & 
  \begin{enumerate}
      \item The system shall allow the staff to generate reports on borrowing and returning activities.
      \item The system shall allow the staff to view analytics on book borrowing trends.
  \end{enumerate} \\
  \hline
  
  Priority & Medium\\
  \hline
  Reference & \\
  \hline
\end{tabular}
\end{center}  



\section{Non Functional Requirements}
\subsection{Usability}
The software should have a clear and user-friendly interface, with intuitive navigation and easy-to-use controls..The software should be consistent in its layout and design, to avoid confusion and make it easier to use.The software should allow for quick and easy data entry and retrieval, reducing the time needed for administrative tasks.The software should have a search function that is easy to use and returns accurate and relevant results.

\subsection{Reliability}
The software should be available for use at all times.The software should have automatic data backups to prevent data loss in case of system failure.The software should be able to recover from errors or crashes without losing any data or corrupting the database.The software should be able to handle concurrent users without performance degradation.The software should be regularly updated and maintained to ensure stability and reliability.

\subsection{Performance}
The system should respond to user input within 2 seconds or less.The system should be able to handle a minimum of 100 concurrent users without experiencing any performance degradation.The system should be able to process book checkout and return transactions within 5 seconds or less.The system should be able to handle at least 10,000 book records without experiencing any slowdown.The system should be able to handle search queries for books by various parameters (author, title, caategory,subcategory) within 3 seconds or less.

\subsection{Supportability}

The system should be easy to install and configure, with clear documentation and installation guides provided to the customer.The system should be modular and easy to maintain, with clear separation of concerns between different components and modules.The system should be designed to minimize downtime and provide automatic backups and recovery options.The system should be designed to work with different hardware and software environments, with clear guidelines and recommendations provided for the customers.

\subsection{Implementation}
The software should be developed using a high-level programming language,  Python, Django and JS.The software should use a database management system (DBMS) to store and retrieve data efficiently. The preferred DBMS for the project is MySQL.The software should be developed using version control software, such as Git, to enable collaboration and manage code changes effectively.
\subsection{Scalability}
The system should be able to handle a minimum of 1000 users concurrently without significant degradation in performance.The response time for critical transactions should not exceed 3 seconds, even under peak load conditions.he database should be designed to handle a large volume of data, with indexing and partitioning strategies to optimize performance.

\subsection{Security}
The system should have a secure login process for all users, including admin and library staff. The login should require a unique username and password for each user.The system should have role-based access control to ensure that users only have access to the functions and data that they are authorized to use.The system should ensure the privacy and confidentiality of user data, such as personal information, borrowing history.The system should have regular backups of all data and implement a disaster recovery plan to ensure data can be recovered in the event of a breach or system failure.The system should maintain an audit trail of all user activities, including logins, transactions, and changes to data, to aid in detecting and investigating security breaches.

\subsection{Testability}
The software must be designed in a modular fashion so that individual components can be easily tested in isolation.Testing and debugging will have to be conducted to ensure maximum reliability of our system.

\subsection{Maintainability}
The product should be modular and well-structured, with clear separation of concerns, to make it easier to understand and modify.
The code should be well-documented to make it easier to maintain and update.The product should be designed with future changes in mind, so that new features and functionality can be added with minimal impact on existing code.

\section{System Model}
\subsection{Scenarios}

\subsubsection{Create new user account}

\begin{itemize}
    \item Admin can create new user account.
    \item Admin gives different information for the account.
    \item Only Admin can access create new user section.
\end{itemize}

\subsubsection{Member Maintenance}

\begin{itemize}
    \item Librarian and staffs who have user account maintain the member of the library.
    \item User can update ,delete the information of the member.
    \item Book lending facilities for the member also maintain by the users.
\end{itemize}

\subsubsection{Book Management}

\begin{itemize}
    \item User can add books according to their section,subsection and categories.
    \item User can filter the books on basis of category.
    \item user can update quantity or availability of the books.
\end{itemize}

\subsubsection{Book Transaction}

\begin{itemize}
    \item Users handle the book transaction section.
    \item Users update the information like status,borrowing date,returning date etc related to each transaction
\end{itemize}

\subsubsection{Book Return}
\begin{itemize}
    \item Users make sure that the book must be returned in the selected date.
    \item After returning book users update the information. 
\end{itemize}



\subsection{Use cases}
\begin{center}

% Create account
\begin{tabular}{ | m{7em} | m{9cm}|} 
  \hline
  Name & Create new account\\ 
  \hline    
  Actor & Admin \\ 
  \hline
  Flow of events & 
    System Admin will create a new account for other librarian and stuff. With the account created , Librarian will have the access to the software. \\
  \hline
  
  Entry Condition & Admin will give information about the new account taken from the library stuff. \\
  \hline
  Exit Condition & The app will create the account, store it in the local database and display it \\
  \hline

\end{tabular}

\vspace{1cm}

\begin{tabular}{ | m{7em} | m{9cm}|} 
  \hline
  Name & Login\\ 
  \hline    
  Actor & Admin , Librarian \\ 
  \hline
  Flow of events & 
    1. User enters email and password
    \newline
    2. User clicks on "Log In" button\\
  \hline
  
   Entry Condition & The user is not logged in and has a registered account  \\
  \hline
  Exit Condition & The user is logged in and taken to the home screen \\
  \hline

\end{tabular}

\vspace{1cm}

\begin{tabular}{ | m{7em} | m{9cm}|} 
  \hline
  Name & Logout\\ 
  \hline    
  Actor &  Admin , Librarian \\ 
  \hline
  Flow of events & 
 User clicks on "Log Out" button \\
  \hline
  
   Entry Condition & The user is logged in \\
  \hline
  Exit Condition & The user is logged out and taken to the login screen \\
  \hline

\end{tabular}

\vspace{1cm}

\begin{tabular}{ | m{7em} | m{9cm}|} 
  \hline
  Name & Update Profile\\ 
  \hline    
  Actor & Admin , Librarian \\ 
  \hline
  Flow of events & 

    1. User selects "Edit Profile" option 
    \newline
    2. User makes desired changes to their profile information
    \newline
    3. User clicks on "Save" button \\
    
  \hline
 
   Entry Condition & The user is logged in .\\
  \hline
  Exit Condition & User's profile information is updated and saved.\\
  \hline

\end{tabular}

\vspace{1cm}

\begin{tabular}{ | m{7em} | m{9cm}|} 
  \hline
  Name & Maintain Book Records\\ 
  \hline    
  Actor & Admin , Librarian \\ 
  \hline
  Flow of events & 

   1. Add Book: Admin/Librarian adds a new book to the library's collection by filling out the book's information in the system.
       \newline
   2.Remove Book: Admin/Librarian removes a book from the library's collection by deleting the book's record from the system.
       \newline
    3. Update Book Info: Admin/Librarian updates a book's information in the system, such as the title, author, or publisher.
        \newline
    4. Add Category/Sub Category: Admin/Librarian adds a new book category to the system. \\
  \hline
  
   Entry Condition & Admin/Librarian is logged in \\
  \hline
  Exit Condition & The book records are updated in the system and the Admin/Librarian can continue to manage the library's book collection.
 \\
  \hline

\end{tabular}

\vspace{1cm}

\begin{tabular}{ | m{7em} | m{9cm}|} 
  \hline
  Name & Issue / Return Book\\ 
  \hline    
  Actor & Admin , Librarian, Member \\ 
  \hline
  Flow of events & 

  1. The librarian or admin searches for the book in the system
  \newline
   2. If the book is available, the librarian or admin processes the borrowing or returning request
     \newline
3.  The system updates the book status accordingly
  \newline
4.  The librarian or admin provides the book to the member (in the case of borrowing) or receives the book from the member (in the case of returning) \\
  \hline
     Entry Condition &
    1. A member requests to borrow or return a book
    \newline
    2.The librarian or admin is logged in and has access to the system\\
  \hline
  Exit Condition & The book is successfully borrowed or returned. \\
  \hline

\end{tabular}

\vspace{1cm}

\begin{tabular}{ | m{7em} | m{9cm}|} 
  \hline
   Name & Maintain Members\\ 
  \hline    
  Actor & Admin , Librarian \\ 
  \hline
  Flow of events & 

    Admin/Librarian adds, removes or updates member information \\
  \hline
  
   Entry Condition & Admin/Librarian is logged in and authorized to maintain member records \\
  \hline
  Exit Condition & Member record is added, removed or updated in the system, and the Admin/Librarian is notified of the action. \\
  \hline

\end{tabular}


\vspace{1cm}

\end{center}


\subsection{Use case diagram}
\begin{figure}[H]
    \centering
    \includegraphics[scale = 0.27]{images/useCase.png}
    \caption{Use Case Diagram}
    \label{fig:use_case_diagram}
\end{figure}

\subsection{Dynamic model}
\subsubsection{Sequence Model}
\begin{figure}[H]
    \centering
    \includegraphics[width=\textwidth]{images/sequence.png}
    \caption{Sequence Diagram}
    \label{fig:sequence_diagram}
\end{figure}

\subsubsection{Statechart model}
\begin{figure}[H]
    \centering
    \includegraphics[width=\textwidth]{images/statechartr.png}
    \caption{Statechart Diagram}
    \label{fig:statechart_diagram}
\end{figure}

\subsection{Activity Diagram}
\begin{figure}[H]
    \centering
    \includegraphics[width=\textwidth]{images/activity.png}
    \caption{Activity Diagram}
    \label{fig:activity_diagram}
\end{figure}
\newpage
\subsection{User interface}

\subsubsection{Home Page}
The Home page shows statistics of all the options for the system which includes Home,Category,Sub Category,Books,Students,Borrowing Transaction,User etc. There will be a menu bar above, which will show the options.

\subsubsection{Menu Bar}
Admin can be able to access all of the sections of menu-bar.But Librarian and other staffs can't access user section part. In this specific section , new user for the software can be added

\subsubsection{Categories}
In this section the books will arrange according to their category.The description and the date from when the book has been added will be provided.Here will be the action part in which the users can update,remove or view the books.Here Users can add the books by filling up the information about the book and the searching option will 
also available in this page.

\subsubsection{Subcategories}
It will be same as categories page.The main deference is it will provide the information more specific for the users.All the options which are available in the Categories page will also be available in this page.

\subsubsection{Books}
In this section books will be displayed with their category/subcategory,title,id added date ,status, action etc.Book searching option will be available in this section.User can add new book ,update the information ,remove and view the book.

\subsubsection{Member}
This section will provide the information of the library member.The information like name,ID,Added date will be available.Users can add or remove the member if they want.There will be update or view option for editing or viewing member's information.Searching option will also available.

\subsubsection{Transaction}
In this section users can add the information of borrowing and returning book.Users will keep the record of the members who will borrow or return the book with their name,id and the book name which will be transacted.

\subsubsection{Create New Account}
Admin can only access this section.Admin can add new account.Other users (Librarian,Stuffs) won't have the access to create new account.


\subsubsection{Software Interface}
The project is a web based application. Hence it will need a working web browser and windows OS. The app can be kept in the pc's memory and can be runned locally.

\subsubsection{Hardware Interface}
The app will need the keyboard/mouse to interfere with the user. 