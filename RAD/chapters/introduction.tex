\chapter{Introduction}
\label{ch:introduction}

\section{Purpose of the System}
The purpose of  E Lib is to streamline and automate the library management process, making it more efficient and accurate. The software is intended to be used by the library staff and the admin, who will have access to a range of tools to manage the library's collection of books and track borrowing and returning activities.

By implementing the E Lib, the library will be able to manage its collection more effectively, resulting in improved member satisfaction and better utilization of resources. The software will also help to reduce the workload of the library staff, allowing them to focus on more important tasks such as member engagement and outreach.

Overall, the E Lib is aligned with the corporate goal of improving operational efficiency and member satisfaction. By providing a modern and efficient library management solution, the library will be better positioned to achieve its objectives and fulfill its mission of serving the needs of its members.




\section{Scope of the System}
The scope of this system includes all the necessary features and functionalities required for managing the library's collection of books and tracking borrowing and returning activities. The software will include tools for user management, book management, borrowing management, returning management, notifications, and reports and analytics. The system will be designed for use by the library staff and admin, who will have access to different levels of functionality based on their role. 

However, it should be noted that the software may be updated or revised in the future to address new requirements or improve functionality. Any changes made to the system will be documented and communicated to all relevant stakeholders.

\section{Objective and success criteria of the project}

The objective of the E Lib project is to improve the efficiency and accuracy of the library management process. The success of the project will be measured by user and member satisfaction, accuracy of records, resource utilization, and adherence to timelines and budget. The specific objectives of the project include providing a user-friendly software tool for managing the library's collection of books, tracking borrowing and returning activities, and improving the experience of library members.





\section{Definitions, Acronyms and Abbreviations}

\begin{itemize}
    \item UI: User Interface

    
    \item API: Application Programming Interface
    
    \item UX: User Experience
    
    \item JS: JavaScript
    
    \item CRUD: Create, Read, Update, Delete
    
    \item HTTPS: Hypertext Transfer Protocol Secure
    
    \item SQL: Structured Query Language
    
    \item JSON: JavaScript Object Notation
\end{itemize}

\section{References}
        [1]. Django: 
        \textbf{\url{https://docs.djangoproject.com/en/4.1/}} \\[1pt]
        [2]. RAD methodology: 
     \textbf{\url{https://en.wikipedia.org/wiki/Rapid_application_development}} \\ [1pt]
      [3]. SQLite: 
        \textbf{\url{https://www.sqlite.org/docs.html}} \\[1pt]
        [4]. Google Cloud Functions: 
        \textbf{\url{ https://cloud.google.com/functions}} \\[1pt]

        
\section{Overview}
E Lib is designed to streamline and automate the library management process, making it more efficient and accurate. The software will be used by the library staff and admin to manage the library's collection of books, track borrowing and returning activities, and improve member satisfaction. The success of the project will be measured by user and member satisfaction, accuracy of records, resource utilization, and adherence to timelines and budget.