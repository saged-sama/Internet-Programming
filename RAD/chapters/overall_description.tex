\chapter{Overall Description}
\label{ch:overall description}

\section{Product perspective}
E-lib is a web application designed to automate library management processes and provide a user-friendly environment for library staff and members. The software is intended to be the central hub for all library operations, allowing for efficient management of book borrowing, returns, and inventory. E-lib is designed to be a standalone system, but can also integrate with other library systems or databases. The software is built using the Python programming language with the Django web framework, and utilizes a SQLite database to store and manage data.

The primary goal of E-lib is to streamline the library management process and reduce the workload on library staff. By automating processes such as cataloging, borrowing, and inventory management, librarians can focus on other important tasks. Additionally, the user-friendly interface of the software provides a seamless experience for library staff and members, making it easy to search for books, check availability, and manage accounts.


\section{Product functions}

The E Lib software has the following major functions:

\begin{itemize}
    \item \textbf{User account management: }  The E lib software enables the admin to create and manage user accounts for library staff. Staff members can then log in and access the system, where they can manage their profile and perform various library management tasks.
    \item \textbf{Member management: }  The software allows library staff to manage member information, including adding new members, editing member details, and deleting members.
    \item \textbf{Cataloging:  } The E lib software provides a cataloging feature that allows staff members to add new books to the system, categorize and subcategorize them, and update book details as needed.
    \item \textbf{Circulation management:}  With the circulation management feature, staff members can manage borrowing transactions, including keeping track of the lend date, due date, and user data.
    \item \textbf{Inventory management:} The software enables staff members to conduct inventory management tasks, such as adding and removing books from the system.

\end{itemize}

\section{User Profiles}
The E lib software is designed to be used by two types of users: admins and library staff.
\begin{itemize}
    \item \textbf{Administrator: }  This user profile will have full access to the system and can perform all functions, including creating accounts for library staff and managing system settings.
    \item \textbf{Library staff: } This user profile will have access to all library management functions, including adding and removing books, managing member accounts, and conducting borrowing transactions.
    \item \textbf{Library member: }  While library members will not have direct access to the system, they can interact with library staff to borrow and return books, as well as request information about their account status or available books.

\end{itemize}

\section{Constraints}
There are no significant constraints that will limit the options available to the developers for the E lib project. However, there are a few factors that could be considered when developing the software:
 \begin{itemize}
    \item \textbf{Technology: }   The software is only available for PC operating systems, specifically Windows. Therefore, the development team should keep in mind the compatibility requirements of the software.
    \item \textbf{Databases:  }SQLite database is used with Python and Django, so the development team should have knowledge and experience in working with these tools.
    \item \textbf{Security: } The software may contain sensitive information, such as user data and borrowing history. Therefore, it is essential to ensure that proper security measures are implemented to protect the data.
 \end{itemize}
 


    
\section{Assumptions and Dependencies}
\textbf{Assumed Factors:}
\begin{itemize}
    \item The software will be developed using modern programming languages and frameworks such as Python, Django etc.
    \item The E Lib software will use SQLite for storing and managing library data.
    \item The system will be accessed via a web interface that is optimized only for desktop.
    \item The software will be tested and deployed in a controlled development environment before finalization.
    

\end{itemize}
\textbf{Dependencies:}
\begin{itemize}
    \item The E-Library software will depend on the modern programming languages, frameworks, and databases that are used to develop it.
    \item Any updates or changes made to the cloud infrastructure provider, programming languages, or databases could affect the software system.

\end{itemize}